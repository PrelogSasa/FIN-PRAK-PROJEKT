\documentclass[a4paper,12 pt]{article}
%\usepackage[slovene]{babel}
\usepackage[utf8]{inputenc}
\usepackage[T1]{fontenc}
\usepackage{lmodern}
\usepackage{graphicx}
\usepackage{amssymb}
\usepackage{hyperref}
\usepackage[top=1.3in, bottom=1.3in, left=1.25in, right=1.25in]{geometry}

\begin{document}
\begin{titlepage}
\begin{center}

\large
Univerza v Ljubljani\\
\normalsize
Fakulteta za matematiko in fiziko\\

\vspace{3 cm} 

\large
Saša Prelog, Žan Jarc\\

\vspace{0.5cm}
\LARGE
\textbf{Grafi z najmanjšim  produktnim ABC indeksom}

\vspace{0.5 cm}
\normalsize
Finančni praktikum



\vspace{3cm}


\vfill

\large Ljubljana, 2019

\end{center}
\end{titlepage}

\newpage

\tableofcontents
\vspace{20mm}

\newpage

\section[Uvod]{Uvod}
Naj bo $T_n = (V, E)$ drevo z $n$ vozlišči in $d_u$ stopnja vozlišča $u \in V(T_n)$. Potem je \textbf{produktni ABC indeks} grafa $T_n$ definiran kot
$$
ABC \Pi (G) = \sqrt{ {\displaystyle \prod_{uv \in E(T_n)}} \frac{d_u + d_v - 2}{d_u d_v}}.
$$
V projektni nalogi bova želela najprej ugotoviti, kateri grafi $T_n$ imajo najmanjši produktni ABC indeks za nek primeren $n \in \mathbb{N}$. Cilj projektne naloge je tudi, da ugotoviva, če imajo najdena minimalna drevesa kakšne podobne lastnosti (unikatnost dreves, urejenost vozlišč s stopnjo >= 3, premer drevesa, ...)

\section[Potek dela]{Potek dela}
Do sedaj sva izračunala vse minimalne produktne ABC indekse za drevesa $T_n$, kjer je bil $n>=15$. Za drevesa s 15 vozlišli je minimalni produktni ABC indeks enak $\frac{25}{589824}$.
Minimalni produktni ABC indeks sva pridobila, tako da sva izračunala vse produktne ABC indekse za drevesa za vsak $n>=15$. V prihodnosti se bova lotila naloge še za večje indekse in še z uporabo genetskih algoritmov.


\end{document}